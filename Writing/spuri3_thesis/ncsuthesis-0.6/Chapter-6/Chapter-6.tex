\chapter{Future Work}
\label{chap-six}
As we have seen in our experimentation the speedup in the run time for the algorithm is dependent of the quality (probability of similarity) of historically available data sets for selection. Better quality selection would lead to better speedups. Thus one future work could involve new methodology for maintaining best quality history data sets.
Another possibility is to explore this History Reuse methodology for other algorithms. The general use of determining best match history data set is generic in nature (i.e. independent of the K-Means algorithm in itself) and thus can be extended to other algorithms such as the \textit{SVD based SVM algorithm}. Only a minor change to the training step specific to K-Means is required in such a case. This thus can be used to prove the comprehensible usability of this algorithm for all algorithms in the same class.
On the systems side the History reuse methodology can be used for memory data validation in Non Volatile Memory (NVM). This would prevent multiple expensive re-writes to NVM from main memory in case of already existing data.

Another area to explore may be the use of non uniform generation of hostogram for a better sample size selection in case of the screening algorithm. We chose not to explore this avenue in current work because our technique for unform buckets was showing significant performance gains. \cite{non_uniform_hist_1} shows an interesting way for generating non uniform histograms while still capturing sufficient desnity information such as to facilitate the selection of the density information more succinctly. Integral Histogram\cite{non_uniform_hist_2} represent a new a way of capturing non uniform histograms in the cartesian space and my be extended to our use as well. Both these above mentioned methods may help with furhter improving the accuracy of the screening algorithm.