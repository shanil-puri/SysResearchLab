\chapter{Conclusion}
\label{chap-seven}

In conclusion we would like to say that probabilistic metric for data-set comparison is
one of the best metrics for this purpose. It tells us the exact degree to which we can
expect the computation reuse to be successful for any historical data-set. It also represents a generic way of ranking data sets based on similarity and may have many more applications than just in history-reuse computation. We have also seen based on the above
results that the above mentioned algorithm is handy for all data sets irrespective if the dimensionality, cardinality or cluster count for the data set.
Though the algorithm shows speedups for all data sets for all cluster indexes we see a substantially higher speedup for larger cluster counts as this results in a speedup in both, the run time of the algorithm as well as the initialization step for the algorithm. The initialization is sped up considerably by our approach and this plays a vital part in the final speedup for larger cluster counts as initialization becomes a significant amount of the run time for both the \textit{Random K-Means} and the \textit{K-Means++} algorithm.
Since this can be used as a drop in replacement for any initialization method, it can be used with any flavour of the original K-Means Algorithm. It preserves the semantic of the original K-means. These appealing properties, plus its simplicity, make it a
practical replacement of the standard K-means as long as viable historical data sets are available for reuse.
